\chapter{Introduction} \label{chap:introduction}
With its continuous developments, the semiconductor industry plays a central role in today's technology-driven society \cite{More1965,Bank1990,Smith1997,Alferov1998,Flamm2003,Wachutka2004,Jenkins2005,Lukasiak2010,Mack2011,Klein2016,Leiserson2020}. In this context, the theoretical understanding of the underlying physical processes (e.\,g.\ from analytical models or numerical simulations) and investigation-based knowledge about their real-world behavior (e.\,g.\ from experimental observations) do not always go hand in hand, which is particularly evident in the course of the progressive miniaturization of semiconductor structures (e.\,g.\ semiconductor nanostructures) in order to increase their efficiency by means of power consumption \cite{Twitchett2002,Beleggia2003,Cooper2006,Cooper2007,Twitchett-Harrison2007,Cooper2009,Somodi2013,Yazdi2015}. A precise knowledge of the electrostatic potential distribution of these kind of real-world devices would open the door to a better understanding of their electrical properties. Perhaps the most established transmission electron microscopic (TEM) and thus imaging technique for the investigation of electrostatic potential distributions in semiconductor structures in the $\si{\nm}$~range is off-axis electron holography (EH) \cite{Lichte1986,Tonomura1987,Cowley1992,DuninBorkowski2004}. Combined with a tomographic approach, EH has demonstrated the increasing influence of TEM-specimens' surface effects (e.\,g.\ the broadening of depletion regions or preparation-related artifacts) with a continuous reduction of the structural dimensions \cite{Twitchett-Harrison2007,Miao2016}.

However, taking this approach to a quantitative level necessitates a comparison with conventional simulation methods (e.\,g.\ finite element method (FEM)), which, due to their nature, require extensive knowledge about the charge carrier distribution and boundary conditions. Besides the extensive computational power required, this knowledge is rarely given for real-world TEM-specimens. This circumstance is addressed in the presented thesis by examining the influence of surface effects of electrically biased TEM-specimens on EH investigations. In particular, the stray fields above and below the TEM-specimen play a significant role in the overall phase modulation of an electron wave, whereby the stray fields depend to a large extent on the electrostatic potential distribution within the TEM-specimen.

So far, the determination of the electrostatic potential distribution in electron beam propagation direction always required tomographic approaches, which entail a high measurement effort (i.\,e.\ tomographic tilt series) \cite{Twitchett-Harrison2007,Scott2012,Wolf2014,Miao2016,Yalisove2021}. This thesis introduces a simple and intuitive model for the approximation of electrostatic potential distributions, capable of modeling the electrostatic potential distribution inside and outside the TEM-specimen with a limited set of parameters. By comparing its calculated projection with the reconstructed measured phase, knowledge about the whole electrostatic potential distribution can be obtained while reducing the measurement effort to only one hologram. The self-developed model uses only independent convolutions of an initial potential distribution with a suitable kernel, allowing the approximation of the entire electrostatic potential distribution of real-world TEM-specimens. By this, a significant reduction of the required computational power as well as a drastically simplified measurement process is achieved, creating a new pathway towards quantitative electron holographic investigations of semiconductor nanostructures.

This thesis includes, besides a small theoretical introduction and a short description of the experimental methods used, a comparison between conventional simulation methods and EH measurements. This is followed by the core of the thesis, in which the self-developed model is presented and evaluated for its robustness. A summary and short outlook on further research conclude the thesis.
