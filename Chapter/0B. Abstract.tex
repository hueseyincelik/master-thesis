\chapter*{Abstract}
The aim of the present thesis is to investigate the influence of surface effects of electrically biased semiconductor nanostructures on electron holographic measurements. A significant influence is played by the stray fields above and below the specimen, which in turn are defined by the electrostatic potential distribution within the specimen (especially in the propagation direction of the electron beam). However, a quantitative description of such ever-smaller devices has so far required tomographic approaches, which entail a significant measurement effort, or modeling by means of established simulation methods, which are highly computationally intensive and require knowledge of the microscopic charge carrier distribution.

These circumstances are addressed here by introducing a simple and intuitive model for the approximation of complex electrostatic potential distributions. The self-developed model uses only independent convolutions of an initial potential distribution with a (normalized) propagation distance dependent 1D~Gaussian kernel and thus allows the reconstruction of the entire electrostatic potential distribution of a real-world specimen from only one measured projection. This enables the approximation of the stray fields above and below as well as the electrostatic potential distribution within the specimen. Through the excellent agreement in comparison with tomographic measurements, the self-developed model was validated. Consequently, a significant reduction of the required computational power as well as a drastically simplified measurement process is enabled, opening up new possibilities for quantitative electron holographic investigations of semiconductor nanostructures.
\begin{otherlanguage}{ngerman}
\chapter*{Deutsche Zusammenfassung}
Ziel der vorliegenden Arbeit ist die Untersuchung des Einflusses von Oberflächeneffekten elektrisch geschalteter Halbleiternanostrukturen auf elektronenholographische Messungen. Einen signifikanten Einfluss spielen hierbei die Streufelder über- und unterhalb der Probe, welche wiederum durch die elektrostatische Potentialverteilung innerhalb der Probe (insbesondere in Ausbreitungsrichtung des Elektronenstrahls) bestimmt sind. Eine quantitative Beschreibung solcher immer kleiner werdenden Bauelemente erforderte jedoch bis jetzt tomographische Ansätze, welche einen signifikanten Messaufwand mit sich führen, oder die Modellierung mittels etablierten Simulationsmethoden, welche höchst rechenintensiv sind und eine Kenntnisse über die mikroskopische Ladungsträgerverteilung erfordern.

Diese Umstände werden hier durch die Einführung eines simplen und intuitiven Modells zur Approximation komplexer elektrostatischer Potentialverteilung angegangen. Das selbstentwickelte Modell verwendet nur unabhängige Faltungen einer anfänglichen Potentialverteilung mit einem (normalisierten) von der Ausbreitungsdistanz abhängigen 1D~Gaußschen Kernel und erlaubt so die Rekonstruktion der gesamten elektrostatischen Potentialverteilung einer realen Probe aus nur einer gemessenen Projektion. Hiermit lassen sich sowohl die Streufelder über- und unterhalb als auch die elektrostatische Potentialverteilung innerhalb der Probe approximieren. Durch die exzellente Übereinstimmung im Vergleich mit tomographischen Messungen konnte das selbstentwickelte Modell validiert werden. Dies ermöglicht eine signifikante Reduktion der benötigten Rechenleistung sowie eine drastische Vereinfachung des Messprozesses und eröffnet neue Möglichkeiten für quantitative elektronenholographische Untersuchungen von Halbleiternanostrukturen.
\end{otherlanguage}
