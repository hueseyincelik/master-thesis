\chapter{Conclusion and Outlook} \label{chap:Conclusion}
The overarching aim of this thesis was to investigate the influence of surface effects of electrically biased semiconductor nanostructures and their stray fields on measuring electrostatic potentials by means of off-axis electron holography. Therefore, both a combined approach, comparing established simulation methods and electron holographic measurements, was pursued and a completely new model for the approximation of complex potential distributions, which was verified by tomographic investigations, was developed.

In \cref{chap:experimental-methods}, the used experimental methods are presented. For this purpose, special automation routines were developed, which are divided into the measurement of the specimens as well as the reconstruction and post-processing of the measurement data, consequently significantly cutting down on measurement time, reducing the required time (excluding TEM alignment) from a full work day to around 30~minutes or less. Based on this automation routine, the comparison between electron holography and the simulations obtained from the different software packages was deepened in \cref{chap:experimental-results}. Here, it was shown that for both TEM-specimens the stray field extends several $\si{\um}$ deep into the vacuum region above and below the TEM-specimen. Additionally, it was revealed that the contribution of the TEM-specimen to the overall phase modulation is limited compared to the stray field (approximately 8\% for the coplanar capacitor and approximately 12\% for the $p$-$p^+$-$n^+$-junction). An excellent agreement between simulation and experimental results was observed for the coplanar capacitor, especially with respect to the expected linear proportionality between externally applied bias voltage and normalized phase slope. For the semiconductor nanostructure, in contrast, strong discrepancies between simulated and measured phases were observed, which are particularly reflected in a strong broadening of the depletion region as well as in a significantly reduced phase jump (both deviations in the range of one order of magnitude). According to the results, these deviations cannot be attributed to the influence of the stray field or the screening layer (even an extended Debye length of $\lambda_D = \SI{18 \pm 5}{\nm}$ has only a negligible influence on the overall phase modulation), but, according to literature \cite{Twitchett-Harrison2007}, originate from preparation-related Ga-ion implantation. This cannot be reproduced with conventional simulation approaches (classical electrostatic approaches require extensive knowledge about the charge carrier distribution and boundary conditions, rarely known for complex TEM-specimens), which makes a new model for the approximation of such potential distributions of real-world semiconductor nanostructures inevitable.

\Cref{chap:SIMP}, which is the core of this thesis, deals with the introduction, systematic characterization as well as the application of such a model to real-world semiconductor nanostructures, verifying its robustness by means of tomographic investigations. The model provides a simple and intuitive approach for modeling complex potential distributions, where a convolution of a initial potential distribution with a (normalized) propagation distance dependent 1D~Gaussian kernel is utilized. Hereby it is able to approximate the potential distribution of the stray field outside the TEM-specimen as well as any distributions inside the TEM-specimen (e.\,g.\ due to preparation-related artifacts) that may arise, requiring minimal knowledge of the TEM-specimen besides an initial electrostatic potential distribution. From a numerical standpoint, the presented model allows for a linear scaling with available computational power through a parallelization of the independent calculations of all intermediate results, which leads to a significant reduction of computation time and required memory (i.\,e.\ seconds instead of days and a few~GBs instead of hundreds of~GBs, respectively). Utilizing the coplanar capacitor as a well known reference specimen, the model was thoroughly tested and characterized for different geometric layouts (i.\,e.\ varying distance between the capacitor plates), resulting in a maximum deviation of $\Delta \phi_{\mathit{max}} = \SI{0.1}{\volt}$ (limited only to an area close to the contacts, where the theoretical approximations made in the model are no longer valid) in a direct comparison with finite element simulations. An excellent agreement between the model and experimental results was shown in particular with tomographic investigations, where the tilt series was limited to an angular range of $\alpha = \pm \SI{34}{\degree}$ in $\SI{2}{\degree}$ increments. Here, the spacing of the capacitor plates was accurately determined to $d_{\mathit{rec}}^{\mathit{rest}} = \SI{480 \pm 20}{\nm}$ and the thickness of the specimen to be $t_{\mathit{rec}}^{\mathit{rest}} = \SI{500 \pm 50}{\nm}$

For a $p$-$p^+$-$n^+$-junction, the self-developed model was extended to a multi-layer framework, consisting of a bulk core, preparation damage and surface layer region (where the preparation damage layer correlates to a region with a FIB-induced changing electrostatic potential distribution in electron beam direction). Through the comparison of CBED and EH measurements, the thickness of the surface was determined to be $t_{\mathit{SL}} = \SI{30 \pm 5}{\nm}$. The bulk core thickness of $t_{\mathit{BC}} = \SI{10}{\nm}$, determined through a comparison with a single reconstructed phase, yields a preparation damage depth of $t_{\mathit{PD}} = \SI{110}{\nm}$. By using these parameters for the approximation of the electrostatic potential distribution, a tomographic comparison (analog to the coplanar capacitor) yielded an excellent agreement between the self-developed model and experimental results. In detail, the specimen thicknesses determined from both methods match within the limits of their measurement uncertainty with $t_{\mathit{rec}}^{\mathit{rest}} = t_{\mathit{rec}}^{\mathit{EH}} = \SI{330 \pm 30}{\nm}$. Moreover, despite the limited resolution in $z$-direction (i.\,e.\ strongly restricted angular range), a significantly reduced width of the potential jump in a central region of the specimen (far from the specimen surfaces) is observed compared to the classical reconstructed phase. The excellent agreement of both tomographic reconstructions validates the self-developed model, where all necessary parameters were obtained from the comparison with only one reconstructed phase. A sufficiently good knowledge of the potential distribution of real-world TEM-specimens (in particular in electron beam direction) can consequently be obtained on the basis of this model with significantly reduced measurement and computational effort.

These findings show that the first major hurdle towards the quantitative investigation of real-world semiconductor nanostructures has been overcome. However, there are still some small hurdles on the way to the finish line. On the one hand, biased TEM-specimens show surface currents and charging effects, which raise the question to what extent a variation of the surface layer in the self-developed model is necessary. On the other hand, a systematic investigation of the influences of FIB parameters, such as the beam current or the acceleration voltage during specimen preparation, is required in order to match them with the adjustable model parameters (e.\,g.\ implantation depth $t_{\mathit{PD}}$). These systematic investigations would go far beyond the scope of this thesis, but provide sufficient basis for further in-depth studies of this highly interesting topic. Nevertheless, this thesis has provided a completely new set of methodological tools for investigating the influence of surface effects of electrically biased semiconductor nanostructures on EH investigations.
